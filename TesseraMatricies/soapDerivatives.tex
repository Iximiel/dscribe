\documentclass{article}
\usepackage{graphicx}
\usepackage{amsmath}
\usepackage[dvipsnames]{xcolor}


\begin{document}

\title{Soap by Tesseral Spherical Harmonis and Derivatives}
\author{Aki Morooka}

\maketitle

\begin{abstract}
This is a documentation on the derivatives of the SOAP spectrum.
\end{abstract}

\section*{Steps}
In short: 

\textcolor{orange}{step 1)} Open real\_spherical\_harmonics.wxmx, and run the function definitions.

\textcolor{orange}{step 2)} run the cell with
\begin{verbatim}
              h[i,j] := rlY(i-1,-(i-j),x,y,z);
              genmatrix(h,5,5*2-1);
\end{verbatim}

where "5" is the l in spherical harmonics, and can be replaced to an arbitrary number.

\textcolor{orange}{step 3)} Copy the matrix as Matlab/Octave, paste it into a plane text file.

\textcolor{orange}{step 4)} Delete char "[" and "]". And replace char ";" to ","

\textcolor{orange}{step 5)} In mat2c.py change the "test.txt" to the file name you used to copy paste the matrix, in line

\begin{verbatim}
        mat = np.loadtxt("test.txt", dtype = "U16384") # Change me
\end{verbatim}

\textcolor{orange}{step 6)} Run python3 mat2c.py, this will produce files tesseral\_mat\_nopow.txt and tesseral\_mat\_nopow.txt.

\textcolor{orange}{step 7)} Finally, run  python3 printMat.py. This will produce the file
\begin{verbatim}
               "finalSoapFunctionsWithoutSqrtPi3.txt"
\end{verbatim}

\textcolor{red} {CAUTION: This will erase the file first.}

\textcolor{orange}{step 8)} Copy paste the list of soap functions to a c or cpp source file. And multiplty them by
\begin{equation}
    %\label{simple_equation}
     \sqrt(\pi)^3
\end{equation}
The only thing missing now is the radial basis, but you only need to multiply the $\alpha$ and $\beta$ terms, and the exponent term.

\textcolor{red} {CAUTION: Do not forget to multiply $\pi^{3/2}$ to the functions.}


\newpage

\section*{Structure of the Final File}
\begin{itemize}
\item the first line is the l=0 term, then the next 3 lines are the l=1 tern, then the next 5 are l=2 terms and so on.

\item the $m$'s run from $m_\text{min}$ to $m_\text{max}$ by 1 step every line startin with new $l$.

\item The $rr=r^2$ and "to the power of" are just numbers after a variable, for example $x2 = x^2$, $rr^2 = r^4$ and so on. The reason for this is to precalculate the powers so it is not done in real time, which would significantly slow down the code if done.

\item There are obvious, and not so obvoius patterns, where you can reuse the computations from before, but in general, just precalculating the power terms would be fast enough.

\item In c or cpp, the $x,y,z$ and ``$rr$ will be the distance from the soap-point to the atoms, which would be looped over i, so it would be x[i], rr[i], x2[i], rr3[i] and so on. You can run 
\begin{verbatim}
                      python3 putIs.py 
\end{verbatim}
to get the file with the [i]'s,
\begin{verbatim}
          "finalSoapFunctionsWithoutSqrtPi3Is.txt"
\end{verbatim}
\end{itemize}

\section*{Derivatives}

\begin{align}
 \frac{\partial\vec{P(x,y,z)}}{\partial x_\delta} &= \sum_m \frac{c_m(x,y,z)c_m(x,y,z)}{\partial x_\delta} \nonumber \\ 
 &= \sum_m 2c_m(x,y,z) \frac{c_m(x,y,z)}{\partial x_\delta}
\end{align}
where $\partial x_\delta$ is the derivative of $x$ at atom $\delta$. 



\end{document}
